\section{GitLab CI im Einsatz als DAST Tool}\label{sec:gitlab-ci-im-einsatz-als-dast-tool}

Die Integration einer DAST Stage in GitLab CI wurde von den Entwicklern so leicht wie möglich gemacht.
Wie bei jedem Sicherheitstest auf einer laufenden Anwendung muss diese laufen.
Dabei kann man im Fall von GitLab stark von Containerization profitieren.
GitLab bietet die Möglichkeit, zuvor gebaute Container in der eigenen Container registry zu speichern und später im DAST Job als sog. Service einzubinden\footnote{Die Grundlagen im Umgang mit Docker und Container Registries werden im Folgenden vorausgesetzt. Nähere Informationen können \href{https://docs.docker.com/get-started/}{hier (Get Started)} und \href{https://docs.docker.com/registry/}{hier (Container Registries)} nachgelesen werden.}.
Diese Services laufen parallel zu dem Container, der für die aktuelle Stage gestartet wird und wird automatisch im selben Docker Netzwerk gestartet, sodass eine einfache Kommunikation zwischen den verschiedenen Containern möglich ist.
Somit wird keine dedizierte Testumgebung benötigt, die bei jedem Einsatz eines DAST Tools wieder zurückgesetzt wird.
Dennoch sind die Ergebnisse vergleichbar mit einem realen Szenario, vorausgesetzt die Anwendung wird später in den gleichen Containern deployed, was heutzutage aber meistens der Fall sein sollte.

\subsection{Nötige Stages und Jobs für die Pipeline}

Da in diesem Teil eine stark reduzierte Pipeline betrachtet wird, die lediglich DAST Funktionalität enthalten soll, werden lediglich zwei Stages und Jobs benötigt.
Die erste Stage enthält das Bauen und Hochladen eines Dockercontainers, der in der zweiten Stage als Service eingebunden werden kann, damit er getestet werden kann

\subsection{Build und Upload eines Dockercontainers in die GitLab Registry}

Der erste Schritt, um die Dockercontainer in GitLab CI bauen zu können ist, Docker im Container zur Verfügung zu stellen.
Dazu wird das offizielle \mintinline{sh}{docker:dind}\footnote{dind steht für \textit{Docker in Docker} und stellt eine und sichere Möglichkeit dar, Docker in einer isolierten Umgebung zu verwenden.} Docker image verwendet, welches als service in die Pipeline eingebunden wird.

Zudem wird der Job in die richtige Stage eingeordnet, was mit \mintinline{yaml}{stage: deploy} geschieht.

Nun stehen alle nötigen Informationen zur Verfügung um Dockerimages bauen und hochladen zu können.
Der erste Schritt hierbei ist die Anmeldung in einer Container Registry, in der das Image hochgeladen werden soll.
Dazu wird in der Pipeline als erstes der Befehl
\begin{minted}{sh}
echo $CI_JOB_TOKEN | docker login -u gitlab-ci-token --password-stdin $CI_REGISTRY
\end{minted}
ausgeführt.

Hierbei werden zwei Umgebungsvariablen verwendet, welche von GitLab CI automatisch zur Verfügung gestellt werden.~\cite{PredefinedVariablesReference}
\begin{itemize}
    \item \textbf{CI\_JOB\_TOKEN:} Dies ist ein einmaliger Token der für den Job generiert wurde und als Passwort in der integrierten Container registry verwendet werden kann, solange der Job läuft.
    \item \textbf{CI\_REGISTRY:} Dies ist die Adresse der integrierten Container Registry von GitLab
\end{itemize}

Im nächsten Schritt wird nun das letzte bestehende docker image aus der registry gepulled.
Dies ist dafür da, um die Möglichkeiten von Docker caches zu nutzen und unnötige komplexe Schritte vermeiden zu können, kann aber auch weggelassen werden, wenn dieses Feature nicht erwünscht ist, oder kein Zeitgewinn dadurch in der Pipeline entsteht.

Nun muss das Dockerfile nur noch mit
\begin{minted}{sh}
docker build --tag $CI_REGISTRY_IMAGE:$CI_COMMIT_SHA --tag $CI_REGISTRY_IMAGE:latest .
\end{minted}

gebaut und getagged werden.
Es empfielt sich die Images zusätzlich zum \textit{latest} Tag auch mit dem Hash des aktuellen Commits zu taggen, um im Fehlerfall stets eine Verbindung zwischen gebautem Docker image und commit herstellen zu können.
Hierfür werden erneut zwei Umgebungsvariablen verwendet:\cite{PredefinedVariablesReference}
\begin{itemize}
    \item \textbf{CI\_REGISTRY\_IMAGE:} Die Adresse der Container Registry für das aktuelle Projekt.
    \item \textbf{CI\_COMMIT\_SHA:} Der Hash des commits auf dem die Pipeline läuft
\end{itemize}

Nun müssen die fertig gebauten Images nur noch mit \mintinline{sh}{docker push} hochgeladen werden.
Die fertige deploy stage sieht nun also so aus:

\begin{minted}{yaml}
deploy:
  services:
  - name: docker:dind
    alias: dind
  image: docker:19.03.5
  stage: deploy
  script:
    - echo $CI_JOB_TOKEN | docker login -u gitlab-ci-token --password-stdin $CI_REGISTRY
    - docker pull $CI_REGISTRY_IMAGE:latest || true
    - docker build --tag $CI_REGISTRY_IMAGE:$CI_COMMIT_SHA --tag $CI_REGISTRY_IMAGE:latest .
    - docker push $CI_REGISTRY_IMAGE:$CI_COMMIT_SHA
    - docker push $CI_REGISTRY_IMAGE:latest

\end{minted}

\subsection{Konfiguration der DAST Stage in GitLab CI}

Der größte Teil der Konfiguration dieser Stage wird von einem GitLab Template eingefügt.\cite{DynamicApplicationSecurity}
Dies ist wie bereits zuvor erwähnt auch der größte Vorteil von GitLab gegenüber anderen Tools.
Das Einfügen des Templates ist denkbar einfach:
\begin{minted}{yaml}
include:
  - template: DAST.gitlab-ci.yml
\end{minted}

Nun muss nur noch das zuvor gebaute Docker Image als service in die Pipeline eingebunden werden:
\begin{minted}{yaml}
services: # use services to link your app container to the dast job
  - name: $CI_REGISTRY_IMAGE:$CI_COMMIT_SHA
    alias: flask_demo
\end{minted}

Nachdem das Template und der Application Service eingefügt wurden lässt sich die restliche Stage über Umgebungsvariablen konfigurieren.
Die dafür nötigen Variablen werden nun im Folgenden nacheinander erklärt.

\subsubsection{Grundlegende Konfiguration}

Mit lediglich drei essentiellen Variablen lässt sich ein vollständiger DAST Prozess in GitLab CI nun konfigurieren.

\paragraph{DAST\_WEBSITE} Diese Umgebungsvariable gibt an, unter welcher Adresse die zu testende Anwendung zu finden ist.
Da das Deployment mit einem Docker Service gemacht wurde, reicht es hier den Hostname des Containers zu verwenden und das passende Protokoll und den passenden Port hinzuzufügen.
\paragraph{DAST\_FULL\_SCAN\_ENABLED} Diese Variable entscheidet darüber, ob der ZAP Baseline Scan\cite{OWASPZAPBaseline} oder der ZAP Full Scan\cite{OWASPZAPFull} verwendet wird.
Der Baseline Scan attackiert die Anwendung nicht aktiv und dauert nur wenige Minuten. Der Full Scan verwendet aktive Methoden und ist damit in der Lage deutlich mehr Sicherheitslücken zu finden.
Dies geschieht auf Kosten der Laufzeit und kann durchaus mehrere Minuten bis Stunden dauern, abhängig von der Größe der zu testenden Software.
\paragraph{DAST\_BROWSER\_SCAN} Mit DAST\_BROWSER\_SCAN aktiviert verwendet GitLab einen Browser basierten Crawler, um ein möglichst realistisches Szenario bereits beim Crawlen der Website zu schaffen.

\subsubsection{Anmeldung}

Die Konfiguration mit den oben genannten Variablen kann allerdings nur Bereiche testen, die öffentlich zugänglich sind, da keine Authentifizerung erfolgt ist.
In vielen Anwendungen würde hierbei also nicht viel mehr als ein \textit{Registrierungs-/Anmeldeformular} und eine \textit{Unauthorized Page} getestet werden.
Daher ist es wichtig, dass der DAST Prozess sich auch in der Anwendung anmelden kann.
Die Konfiguration dieser Anmeldung ist ebenfalls denkbar einfach und wird wieder über Umgebungsvariablen gesteuert.
Diese werden nun im Folgenden beschrieben und schließen die Konfiguration der DAST stage ab.

\paragraph{DAST\_AUTH\_URL}
\paragraph{DAS\_USERNAME}
\paragraph{DAST\_PASSWORD}
\paragraph{DAST\_USERNAME\_FIELD}
\paragraph{DAST\_PASSWORD\_FIELD}
\paragraph{DAST\_SUBMIT\_FIELD}
\paragraph{DAST\_EXCLUDE\_URLS}
\paragraph{DAST\_AUTH\_VERIFICATION\_URL}
\paragraph{DAST\_AUTH\_VERIFICATION\_SELECTOR}
\paragraph{DAST\_AUTH\_REPORT}

\subsection{Vollständige DAST Konfiguration für GitLab CI}

Kombiniert sieht die fertige vollständige Konfiguration für eine DAST Pipeline in GitLab CI also wie Folgt aus:
\begin{minted}{yaml}
stages:
  - deploy
  - dast

include:
  - template: DAST.gitlab-ci.yml

variables:
  DAST_WEBSITE: http://flask_demo:5000
  DAST_AUTH_URL: $DAST_WEBSITE/login
  DAST_USERNAME: "example@mail.com"
  DAST_PASSWORD: "password" # use custom, masked GitLab variable here
  # a selector describing the element containing the username field at the sign-in HTML form
  DAST_USERNAME_FIELD: "id:email"
  # a selector describing the element containing the password field at the sign-in HTML form
  DAST_PASSWORD_FIELD: "id:password"
  # the selector of the element that when clicked will submit the login form or
  # the password form of a multi-page login process
  DAST_SUBMIT_FIELD: "css:button[type='submit']"
  # optional, URLs to skip during the authenticated scan; comma-separated,
  # no spaces in between
  DAST_EXCLUDE_URLS: "$DAST_WEBSITE/logout"
  # optional, used to verify authentication is successful by expecting this URL once the
  # login form has been submitted
  DAST_AUTH_VERIFICATION_URL: "$DAST_WEBSITE/profile"
  # optional, used to verify authentication is successful by expecting a selector to be
  #present on the page once the login form has been submitted
  DAST_AUTH_VERIFICATION_SELECTOR: "id:greeting"
  # optionally output an authentication debug report
  DAST_AUTH_REPORT: "true"
  # do a full scan, including active attacks
  DAST_FULL_SCAN_ENABLED: "true"
  DAST_BROWSER_SCAN: "true"

services: # use services to link your app container to the dast job
  - name: $CI_REGISTRY_IMAGE:$CI_COMMIT_SHA
    alias: flask_demo

deploy:
  services:
  - name: docker:dind
    alias: dind
  image: docker:19.03.5
  stage: deploy
  script:
    - echo $CI_JOB_TOKEN | docker login -u gitlab-ci-token --password-stdin $CI_REGISTRY
    - docker pull $CI_REGISTRY_IMAGE:latest || true
    - docker build --tag $CI_REGISTRY_IMAGE:$CI_COMMIT_SHA --tag $CI_REGISTRY_IMAGE:latest .
    - docker push $CI_REGISTRY_IMAGE:$CI_COMMIT_SHA
    - docker push $CI_REGISTRY_IMAGE:latest
\end{minted}
