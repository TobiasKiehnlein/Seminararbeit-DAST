\section{Vergleich zwischen Burp Suite, OWASP ZAP und GitLab für den Einsatz im DAST}\label{sec:vergleich-zwischen-burp-suite-owasp-zap-und-gitlab-für-den-einsatz-im-dast}
\subsection{Burp Suite}

\begin{quote}
    Burp Suite Professional is the web security tester's toolkit of choice.
    Use it to automate repetitive testing tasks - then dig deeper with its expert-designed manual and semi-automated security testing tools.
    Burp Suite Professional can help you to test for OWASP Top 10 vulnerabilities - as well as the very latest hacking techniques.\cite{BurpSuiteProfessional}
\end{quote}

Mit diesen Worten beschreibt Portswigger deren Tool Burp Suite, eines der größten, bekanntesten und ohne Frage besten Frameworks, wenn es um das Thema Anwendungssicherheit geht.
Allerdings kommt mit großer Popularität und großer Funktionalität meistens auch ein großer Preisfaktor ins Spiel.
Bei Burp Suite wird hierbei in zwei unterschiedliche Preismodelle untergliedert.
Burp Suite Community Edition und Burp Suite Professional.
Burp Suite Community Edition ist kostenfrei, während Burp Suite Professional für 349€ pro Jahr zu erwerben ist.

Burp Suite bietet alle Features, die man von einem Tool zum dynamic application security testing erwarten kann und sollte.
Besonders relevant sind hier die in den OWASP TOP 10\cite{OWASPTopTen} gelisteten Sicherheitslücken, welche alle von Burp Suite gefunden werden können.

\subsection{OWASP ZAP}

\begin{quote}
The world’s most widely used web app scanner.
Free and open source.
Actively maintained by a dedicated international team of volunteers.\cite{ZAPHomepage}
\end{quote}

Diese Beschreibung von ZAP bringt es gut auf den Punkt.
ZAP ist ein Open Source Projekt, welches von vielen Entwicklern weltweit kostenlos zur Verfügung gestellt wird.
Daraus entstehen auch durchaus Nachteile, wie zum Beispiel eine relativ kleine Dokumentation, vor Allem im Vergleich zu Burp Suite.
Das Basisfeatureset ist bei OWASP ZAP allerdings nahezu identisch, wie bei Burp Suite.
Dennoch ist Burp Suite in der Erkennung einiger Sicherheitslücken, sowie der Abdeckung durchaus noch überlegen gegenüber dem Community Tool ZAP.\cite{teamBurpSuiteVs2021}

\subsection{GitLab}

GitLab selbst unterstützt keinerlei DAST Funktionalität.
Allerdings stellt GitLab ab dem Ultimate Tier eine nahtlose Integration zu OWASP ZAP zur Verfügung.
Diese ermöglicht es die volle Funktionalität von ZAP in wenigen Zeilen in eine GitLab CI Pipeline zu integrieren.
Hierin liegt auch die Stärke von GitLab im Bereich DAST.
Bei der Konfiguration von Tools wie Burp Suite oder ZAP werden oft Experten benötigt, was mit einem hohen Kostenaufwand verbunden ist.
GitLab CI ermöglicht es mit geringen Mehrkosten eine solche Pipeline aufzusetzen, zu konfigurieren und eine hohe Application Security gewährleisten zu können.
Allerdings ist GitLab Ultimate mit \$1188 USD pro Jahr pro Nutzer ebenfalls kein billiges Tool.
Aufgrund der vielseitigkeit setzen bereits viele Unternehmen auf GitLab aus anderen Gründen und können daher GitLabs DAST Anbindung mehrkostenfrei nutzen.
Besonders in diesem Fall empfiehlt sich die Verwendung von GitLab CI in einer DAST Pipeline.

\subsection{Fazit}

Obwohl Burp Suite einen geringfügig größeren Funktionsumfang hat bietet ZAP mit einer Open Source Lösung eine würdige Alternative.
Mit der Einfachheit in der Konfiguration von GitLab kann allerdings keines der beiden Tools mithalten.
Da in dem Seminar bereits mit GitLab gearbeitet wurde hat sich die Verwendung der integrierten Tools offensichtlich angeboten, weshalb die Entscheidung bei der Auswahl des Tools auf GitLab CI gefallen ist.