% Kurze Einleitung und Erklärung der Wichtigkeit des Themas und des Spezialgebiets.

\section{Einleitung}\label{sec:introduction}
Mit der Verbreitung agiler Methoden und moderner Softwareentwicklung wird stets das Ziel verfolgt in kurzen Abständen Änderungen am Code vorzunehmen und diesen zu veröffentlichen.
Nicht selten ist die Veröffentlichung einer Software ein fehleranfälliger und/oder komplexer Prozess, was zur Folge hat, dass Release-Zyklen länger werden, Deployments eine unangenehme Last werden und Sicherheit oftmals völlig außer Acht gelassen wird.

Gerade deshalb sollte man sich über die Prozesse Gedanken machen, die nach der Implementation des Codes stehen.
Dies stellt allerdings keine triviale Aufgabe dar, weshalb sich die Frage stellt, warum dieser Aufwand in Kauf genommen werden sollte.
Auch wenn hier viele Faktoren eine Rolle spielen, wir sich diese Arbeit auf die Anwendungssicherheit fokussieren.
Nicht selten haben Entwickler keine oder nur wenig Kenntnisse im Bereich der IT-Sicherheit.
Nicht grundlos befinden sich seit Jahren bekannte Sicherheitslücken wie SQL-Injections, XSS oder XSRF immer noch unter den häufigsten Sicherheitslücken im Web.\cite{invictiInvictiAppSecIndicator2021}

Wie lässt sich nun ein derart fundamentales Problem in der IT Industrie lösen?
Um eine Software bestmöglich vor Sicherheitslücken zu schützen, bleibt nur die Option, diese regelmäßig auf Schwachstellen zu untersuchen, um gerade häufigen und leicht zu findenden Sicherheitslücken vorbeugen zu können.
Dies bringt allerdings einen hohen Kostenfaktor in das Projekt, da Sicherheitsexperten zumeist relativ teuer sind.
Dies wird dadurch verstärkt, dass einige triviale Tests wiederholt durchgeführt werden müssen, da kleinste Codeänderungen schnell große Auswirkungen im gesamten Projekt haben können.

Eine einfache Lösung stellen daher automatisierte Sicherheitstests dar, welche die Software auf die gängigsten und bekanntesten Lücken prüft.
Diese können daraufhin direktes Feedback an Entwickler und Sicherheitsexperten geben, um Entwickler auf mögliche Sicherheitslücken zu sensibilisieren und Sicherheitsexperten um repetitive Arbeit zu entlasten.
Somit können Zeit, Geld und Nerven gespart und eine sichere und effiziente Softwareentwicklung gewährleistet werden.