% Kurze Einleitung und Erklärung der Wichtigkeit des Themas und des Spezialgebiets.

\section{Einleitung}\label{sec:introduction}
Mit der Verbreitung agiler Methoden und moderner Softwareentwicklung wird stets das Ziel verfolgt in kurzen Abständen Änderungen am Code vorzunehmen und diesen zu veröffentlichen.
Nicht selten ist die Veröffentlichung einer Software ein fehleranfälliger und/oder komplexer Prozess, was zur Folge hat, dass Release-Zyklen länger werden, Deployments eine unangenehme Last werden und Sicherheit oftmals völlig außer Acht gelassen wird.

Gerade deshalb sollte man sich über die Prozesse Gedanken machen, die nach der Implementation des Codes stehen.
Wie sich allerdings herausstellt, ist diese Aufgabe nicht trivial.
Warum sollte also dieser Aufwand getrieben werden?
Hier spielen viele Faktoren eine Rolle, allerdings wird sich diese Arbeit primär auf den Aspekt der Anwendungssicherheit fokussieren.
Nicht selten haben Entwickler keine oder nur wenig Kenntnisse im Bereich der IT-Sicherheit.
Nicht grundlos befinden sich seit Jahren bekannte Sicherheitslücken wie SQL-Injections, XSS oder XSRF immer noch unter den häufigsten Sicherheitslücken im Web.\cite{invictiInvictiAppSecIndicator2021}

Wie lässt sich nun ein derart fundamentales Problem in der IT Industrie lösen?
Um eine Software bestmöglich vor Sicherheitslücken zu schützen, bleibt nur die Option, diese regelmäßig und häufig auf Schwachstellen zu untersuchen, um gerade häufigen und leicht zu findenden Sicherheitslücken vorbeugen zu können.
Dies bringt allerdings einen hohen Kostenfaktor in das Projekt, da Sicherheitsexperten zumeist relativ teuer sind.
Vor allem unter dem Gesichtspunkt, dass einige triviale Tests wiederholt durchgeführt werden müssen, da Codeänderungen große Auswirkungen im gesamten Projekt haben können.

Die Lösung scheint also einfach: Mit automatisierten Sicherheitstest, welche die Software auf die gängigsten und bekanntesten Lücken prüft und den Entwicklern und Sicherheitsexperten direktes Feedback gibt, um Entwickler auf mögliche Sicherheitslücken zu sensibilisieren und Sicherheitsexperten um repetitive Arbeit zu entlasten und somit auch Geld und Zeit zu sparen.
So kann eine sichere und effiziente Softwareentwicklung gewährleistet werden.